\documentclass[norsk,a4paper,12pt]{article}
\usepackage[T1]{fontenc} %for å bruke æøå
\usepackage[utf8]{inputenc}
\usepackage{graphicx} %for å inkludere grafikk
\usepackage{verbatim} %for å inkludere filer med tegn LaTeX ikke liker
\usepackage{mathpazo}
\usepackage{amsmath}
\usepackage{float}
\usepackage{subcaption}
\title{Project 1}
\author{Richard Andre Fauli}
\date{\today}
\begin{document}

\maketitle

\begin{abstract}
\end{abstract}

\section{Introduction}
\section{Theory}
%The three-dimensional Poisson equation can be given by
%\begin{equation}
%\nabla ^2 \Phi = - 4\pi \rho (\textbf{r}),
%\label{eq:3d_Poisson}
%\end{equation}
%where $\Phi$ is the electrostatic field from a localized charge distribution $\rho(\textbf{r})$. Equation (\ref{eq:3d_Poisson}) can be simplified to one dimension by assuming 

The equation
\begin{equation}
-u''(x) = f(x),
\label{eq:udereqf}
\end{equation}
from $x=0$ to $x=1$ with Dirichlet boundary conditions ($u(0)=u(1)=0$), where $f(x)$ is known, can be solved numerically. To find an approximation for $-u''(x)$, we can use that
\begin{equation}
-\frac{v_{i+1}+v_{i-1} - 2v_i}{h^2} \approx f_i\, , \text{ for } i = 1, 2, ..., n,
\label{eq:uderapprox}
\end{equation}
where $f_i=f(x_i)$, $x_i = ih$ up to $x_{n+1} = 1$, $h$ is the step length given by $h=1/(n+1)$, and $v_i$ is the approximation to the second derivative in point $x_i$. 

To solve equation (\ref{eq:uderapprox}), we can rewrite it as a linear set of equations on the form \begin{equation}
A\textbf{v} = \tilde{\textbf{b}},
\label{eq:Aveqb}
\end{equation} For $i=1$, we have that 
$$h^2f_1 \approx -v_0 + 2v_1 - v_2 = 2v_1-v_2,$$
and for $i=n$ we have 
$$h^2f_n = -v_{n-1} + 2v_n - v_{n+1} = -v_{n-1} + 2v_n.$$ From the equations for the cases $i=1,...\,,i,...\,,n$, $A$ must be a tridiagonal matrix with $-1$ and $2$ along the diagonals
$$A = \begin{pmatrix}
2&-1&0&0&...&0&0&0\\
-1&2&-1&0&...&0&0&0\\
0&-1&2&-1&...&0&0&0\\
:&:&:&:& : &:&:&:\\
0&0&0&0&...&-1&2&-1\\
0&0&0&0&...&0&-1&2
\end{pmatrix}$$
and $\tilde{\textbf{b}}_i = h^2f_i$
\section{Method}
\subsection{Gaussian elimination}
To solve equation (\ref{eq:Aveqb}) using Gaussian elimination, we first look at the general case where $$\tilde{
b}_i = a_i\,v_{i-1} + b_i\,v_i + c_i\,v_{i+1},$$ where $a_1=0$ and $c_n=0$.
For $n=4$, $A$ is a $4$ x $4$ matrix:
$$A = \begin{pmatrix}
b_1&c_1&0&0\\
a_1&b_2&c_2&0\\
0&a_2&b_3&c_3\\
0&0&a_3&b_4
\end{pmatrix}.
$$
To solve equation (\ref{eq:Aveqb}) where $A$ is this $4$ x $4$ matrix, we first do a round of forward substitutions.
$$\begin{pmatrix}
b_1&c_1&0&0&\tilde{b}_1\\
a_1&b_2&c_2&0&\tilde{b}_2\\
0&a_2&b_3&c_3&\tilde{b}_3\\
0&0&a_3&b_4&\tilde{b}_4
\end{pmatrix} \rightarrow \begin{pmatrix}
b_1&c_1&0&0&\tilde{b}_1\\
0&b_2-\frac{a_1}{b_1}c_1&c_2&0&\tilde{b}_2-\tilde{b}_1\frac{a_1}{b_1}\\
0&a_2&b_3&c_3&\tilde{b}_3\\
0&0&a_3&b_4&\tilde{b}_4
\end{pmatrix}.$$
We then define $d_1=b_1$, $d_2=b_2-(a_1/b_1)c_1$ and $k_1=\tilde{b}_1$, $k_2=\tilde{b}_2-\tilde{b}_1(a_1/b_1)$. 
$$\begin{pmatrix}
d_1&c_1&0&0&k_1\\
0&d_2&c_2&0&k_2\\
0&a_2&b_3&c_3&\tilde{b}_3\\
0&0&a_3&b_4&\tilde{b}_4
\end{pmatrix} \rightarrow \begin{pmatrix}
d_1&c_1&0&0&k_1\\
0&d_2&c_2&0&k_2\\
0&0&b_3-\frac{a_2}{d_2}c_2&c_3&\tilde{b}_3-\frac{a_2}{d_2}k_2\\
0&0&a_3&b_4&\tilde{b}_4
\end{pmatrix}$$
\section{Results}
\section{Discussion}
\section{Conclusion}
\section{References}
\section{Code attachment}
\end{document}